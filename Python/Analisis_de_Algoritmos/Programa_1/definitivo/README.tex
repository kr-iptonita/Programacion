\documentclass{article}
\usepackage{graphicx} % Required for inserting images
\usepackage{babel}
\title{README.txt}
\author{kr\_ipton\\
Juárez Torres Karla Romina}
\date{Febrero 2023}

\begin{document}

\maketitle

\section{README}
El código proporcionado genera una cuadrícula de adoquines mediante la técnica de la recursión.
La cuadrícula se crea a partir de una matriz de tamaño especificado por el usuario, se elige aleatoriamente una posición en la matriz y se asigna el valor de $-1$ a esa posición,
que será el cuadro especial. Luego, se llama a la función \textsc{recursiva\_ adoquín }
para rellenar la cuadrícula con números aleatorios que se asignan a cada posición en la matriz. Finalmente, se imprime la cuadrícula en la terminal y se define una lista de colores aleatorios y
un diccionario que asocia cada número de la cuadrícula con un color.\\

La función \textsc{recursiva\_ adoquín } es la función principal del código y es responsable de dividir la cuadrícula en secciones más pequeñas y
colocar bloques de adoquines en las secciones adecuadas. La función toma tres argumentos: n, que es el tamaño de la sección actual, x e y,
que son las coordenadas de la esquina superior izquierda de la sección actual en la matriz. La función utiliza una variable global llamada "centro"
para asignar un número único a cada bloque de adoquín que se coloca.\\

La función \textsc{recursiva\_ adoquin } comienza verificando si la sección actual es lo suficientemente pequeña como para colocar los bloques de adoquines directamente en la matriz.
Si es así, la función coloca los bloques en la matriz y devuelve 0.\\

Si la sección actual no es lo suficientemente pequeña, la función busca la posición del bloque especial en la sección actual.
Si no hay un bloque especial en la sección actual, se divide la sección en cuatro secciones más pequeñas y se llama recursivamente a la función para cada sección.
Si hay un bloque especial en la sección actual, se coloca un bloque de adoquín en la sección adecuada y se llama recursivamente a la función para cada sección que no contiene el bloque especial.\\

La colocación de los bloques de adoquines se realiza en la función colocación. La función toma seis argumentos: \textsc{x\_1, y\_1, x\_2, y\_2, x\_3, y\_3 },
que son las coordenadas de las tres posiciones en la matriz donde se colocarán los bloques de adoquines. La función utiliza la variable global "centro"
para asignar el mismo número a los tres bloques de adoquines.\\

El código también define una función \textsc{try-except } que solicita al usuario que ingrese un valor entero para el tamaño de la cuadrícula.
Si el usuario ingresa un valor que no es un número entero, el programa termina con un mensaje de error. Si el usuario ingresa un valor entero,
el programa continúa y crea una matriz de ceros con el tamaño especificado de la cuadrícula.\\

En general, el código utiliza técnicas de programación recursiva y aleatoriedad para generar una cuadrícula de adoquines.

\end{document}
